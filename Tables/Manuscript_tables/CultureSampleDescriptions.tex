% latex table generated in R 4.0.2 by xtable 1.8-4 package
% Fri Jul 24 12:57:00 2020
\begin{table}[ht]
\centering
\begin{tabular}{lllrl}
  \hline
Broad taxon & Species & Strain & \textit{n} & Short ID \\ 
  \hline
Archaea & \textit{Nitrosopumilus maritimus} & SCM1 &   3 & Nmar \\ 
  Cyanobacteria & \textit{Crocosphaera watsonii} & 8501 &   3 & 8501 \\ 
  Cyanobacteria & \textit{Prochlorococcus marinus} & 1314 &   3 & 1314P \\ 
  Cyanobacteria & \textit{Prochlorococcus marinus} & AS9601 &   3 & As9601 \\ 
  Cyanobacteria & \textit{Prochlorococcus marinus} & MED4 &   3 & MED4 \\ 
  Cyanobacteria & \textit{Prochlorococcus marinus} & NATL2A &   3 & Nat \\ 
  Cyanobacteria & \textit{Synechococcus sp.} & 7803 &   2 & 7803 \\ 
  Cyanobacteria & \textit{Synechococcus sp.} & 8102 &   2 & 8102 \\ 
  Diatom & \textit{Cyclotella meneghiniana} & 338 &   3 & Cy \\ 
  Diatom & \textit{Navicula pelliculosa} & 543 &   3 & Np \\ 
  Diatom & \textit{Phaeodactylum tricornutum} & 2561 &   2 & Pt \\ 
  Diatom & \textit{Pseudo-nitzschia pungens} & Pc55x &   3 & Pc55x \\ 
  Diatom & \textit{Thalassiosira oceanica} & 1005 &   3 & To \\ 
  Diatom & \textit{Thalassiosira pseudonana} & 1335 &   3 & Tp \\ 
  Dinoflagellate & \textit{Alexandrium tamarense} & 1771 &   3 & 1771 \\ 
  Dinoflagellate & \textit{Amphidinium carterae} & 1314 &   3 & 1314 \\ 
  Dinoflagellate & \textit{Heterocapsa triquetra} & 449 &   3 & 449 \\ 
  Dinoflagellate & \textit{Lingulodinium polyedra} & 2021 &   3 & 2021 \\ 
  Haptophyte & \textit{Emiliania huxleyi} & 2090 &   3 & 2090 \\ 
  Haptophyte & \textit{Emiliania huxleyi} & 371 &   3 & 371 \\ 
  Prasinophyte & \textit{Micromonas pusilla} & 1545 &   3 & 1545 \\ 
  Prasinophyte & \textit{Ostreococcus lucimarinus} & 3430 &   3 & 3430 \\ 
   \hline
\end{tabular}
\caption{\label{CultureSampleDescriptions}Summary of cultured organisms analyzed in this study. More information (including culturing conditions for all except the Archaea) can be found in \cite{Durham2019}. More detailed information are in Table 
ef{FullCultureSampleDescriptions}. Short ID is how the organism is labeled throughout the figures.} 
\end{table}
